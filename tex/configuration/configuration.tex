\documentclass[11pt, oneside]{book}
%------------------------
%------- PAQUETES -------
%------------------------

% Establecemos el tamaño de página y márgenes
\usepackage[a4paper, total={6in, 8in}]{geometry}

% Establecemos la distancia entre párrafos
\setlength{\parskip}{1em}

\usepackage[utf8]{inputenc}
\usepackage[british, spanish]{babel}
\usepackage[T1]{fontenc}

% Bibliografía
\usepackage[backend=bibtex,style=authoryear,citestyle=authoryear]{biblatex}
\usepackage{csquotes}

% Visualización de marcadores en el lector PDF
\usepackage{bookmark}

% \overset, numeración de teoremas y otras
\usepackage{amsmath}

% Para cambiar numeración de "enumerate"
\usepackage{enumitem}

% Para \mathbb etc.
\usepackage{amssymb}

% Para inserción de imágenes
\usepackage{graphicx}

% Para definición de teoremas
\usepackage{amsthm}

% Para gráficos
\usepackage{tikz}

%------------------------
%------- TEOREMAS -------
%------------------------

\newtheorem{thm}{Teorema}[chapter] % Estilo de texto cursivo

\theoremstyle{definition} % Cambio del estilo de los teoremas a normal
\newtheorem{proposition}[thm]{Proposición}
\newtheorem{definition}[thm]{Definición}
\newtheorem{corolario}[thm]{Corolario}
\newtheorem{example}[thm]{Ejemplo}
\newtheorem{observation}[thm]{Observación}
\newtheorem*{lema}{Lema}
\newtheorem*{nota}{Nota}

%------------------------
%------- COMANDOS -------
%------------------------

% Conjuntos y constantes matemáticas
\newcommand{\R}{\mathbb{R}} % Reales
\newcommand{\C}{\mathbb{C}} % Complejos
\newcommand{\K}{\mathbb{K}} % Cuerpo K (reales o complejos)
\newcommand{\Sp}{\mathbb{S}} % Esfera
\newcommand{\e}{\mathrm{e}} % número e
\newcommand{\N}{\mathbb{N}} % Naturales
\newcommand{\Q}{\mathbb{Q}} % Racionales
\newcommand{\B}{\mathcal{B}} % Base

\newcommand{\qsh}{\textsf{Q}\texttt{\#}} % Q#
\newcommand{\csh}{\textsf{C}\texttt{\#}} % C#

\newcommand{\orden}[1]{\mathcal{O}\left(#1\right)}

\newcommand{\oversim}[1]{\overset{_\sim}{#1}} % Para poner ~ sobre algo

% Para poner datos encima y/o debajo de implica
\newcommand{\ximplies}[2]{\underset{#2}{\overset{#1}\implies}}
\newcommand{\xiff}[2]{\underset{#2}{\overset{#1}\iff}}
\newcommand{\ximpliedby}[2]{\underset{#2}{\overset{#1}\impliedby}}

% Producto escalar y norma
\newcommand{\dotproduct}[2]{\langle#1,#2\rangle}
\newcommand{\norm}[1]{\left|\left|#1\right|\right|}

% Notación de Dirac -- SUSTITUIR POR PAQUETE DE LUIS
\newcommand{\ket}[1]{\left|#1\right\rangle}
\newcommand{\bra}[1]{\left\langle#1\right|}
\newcommand{\braket}[2]{\left\langle#1|#2\right\rangle}

% Vectores
\newcommand{\twovector}[2]{\begin{pmatrix} #1 \\ #2 \end{pmatrix}} % Vector de dim 2

% Funciones e info debajo de funciones
\newcommand{\function}[3]{#1\colon #2\longrightarrow #3}
\newcommand{\xfunction}[4]{\underset{#4}{{#1\colon #2\longrightarrow #3}}}

% Introducir información debajo de igualdad
\newcommand{\equals}[2]{\underset{#2}{\underset{\shortparallel}{#1}}}

% Comandos para palabras dentro de ecuaciones $ $ o \[ \]
\newcommand{\y}{\mathrm{\ y\ }}
\newcommand{\cte}{\mathrm{cte}}
\newcommand{\rg}{\mathrm{rg}}
\newcommand{\id}{\mathrm{id}}

%Página en blanco NO numerada
\newcommand\blankpage{%
    \null
    \thispagestyle{empty}%
    \addtocounter{page}{-1}%
    \newpage}


\title{Computación cuántica: Pruebas de mutación}
\author{Luis Aguirre \& Javier Pellejero}