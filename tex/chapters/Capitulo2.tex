\chapter{Introducción a la computación cuántica}

En este capítulo vamos a introducirnos en el mundo de la computación cuántica. No nos adentraremos en la teoría de la mecánica cuántica y usaremos nociones básicas de matemáticas. Existen multitud de fuentes que ahondan en el conocimiento aportado por las matemáticas, la física y las ciencias de la computación para cimentar la teoría que vamos a desarrollar a continuación. Nos remitimos a ellas si existe el deseo de conocer más sobre esta rama de la ciencia o incluso a cualquiera de los TFG de los miembros de este grupo presentados para el grado de matemáticas.

Antes de empezar, vamos a hablar sobre una notación muy usada en mecánica cuántica y que emplearemos a menudo en este y los siguientes capítulos.

\section{Notación de Dirac}

La notación $\ket{\psi}$ denominada \textit{ket} pertenece a la notación de Dirac y representa al vector $\psi$ de cierto espacio vectorial complejo como columna, mientas que $\bra{\psi}$ representa al \textbf{conjugado} de $\psi$ como una fila. Por tanto, $\bra{\psi'}\ket\psi$ o $\braket{\psi'}{\psi}$ denota el \textbf{producto escalar complejo} dado por

\begin{equation}
\dotproduct\psi{\psi'}=\sum_{i=1}^{n}\psi_i\overline{\psi'_i}
\end{equation}

Denotamos $\ket\psi\bra\psi$ como el \textbf{producto exterior}. Por la definición anterior dadas para \textit{bra} y \textit{ket} se trata de una matriz de dimensiones $n\times n$ donde $n$ es la dimensión del espacio complejo donde habita $\psi$. Al tratarse de una matriz cuadrada, podemos identificarla como la matriz asociada de un isomorfismo $\C^n\to\C^n$.

\section{Qubit}

El \textbf{qubit} o \textbf{cúbit} es el sistema de información más básico de la computación cuántica. Se trata de un vector unitario de del espacio vectorial $\C^2$ con una base ortonormal prefijada que denotamos por $\{\ket0,\ket1\}$. A menudo, a estos vectores ortonormales se les identifica con dos vectores de $\C^2$, habitualmente $\twovector{1}{0}$ y $\twovector{0}{1}$.