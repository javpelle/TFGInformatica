\chapter*{}

\section*{Resumen (español)}
A lo largo de la última década, la \emph{computación cuántica} ha ido ganando protagonismo en el campo científico. Con ello, han surgido multitud de herramientas enfocadas a este nuevo area y,  entre ellas, lenguajes de programación de naturaleza cuántica.

A su vez, en una sociedad tan tecnológica como lo es la nuestra, la comprobación del correcto funcionamiento del software desarrollado se ha convertido en una necesidad imperante. De esta forma aparece todo un ámbito de la Ingeniería del Software dedicada a cubrir esta necesidad: \emph{Testing de Software}.

Así, surge la idea de tomar una de las herramientas más importantes del testing, el \textit{mutation testing}, y tratar de trasladarla al mundo cuántico para aplicarla a dos de los lenguajes de computación cuántica más relevantes en el momento: $\qsh$  y \textit{Qiskit}.

Para ello, se ha desarrollado un programa en Java que permite tomar código de los dos lenguajes mencionados previamente y aplicar las técnicas más importantes del \textit{mutation testing}. Además, si bien en la versión actual sólo se puede ejecutar código de $\qsh$ y \textit{Qiskit}, este programa ha sido desarrollado bajo la premisa de poder añadir otros lenguajes cuánticos en el futuro.

A lo largo de este trabajo se explicará con detalle las nociones más importantes de la computación cuántica, así como de la disciplina del testing, haciendo especial énfasis en las pruebas de mutación. También se detalla la planificación seguida para la fase de desarrollo, así como la estructura y el funcionamiento interno del programa. Por último se muestra la ejecución del programa para uno de los algoritmos cuánticos más importantes: \textit{el algortimo de Deutsch-Jozsa.}

\textbf{Palabras Clave}: Computación cuántica, \textit{testing} cuántico, pruebas de mutación, Microsoft \qsh, IBM Qiskit, herramienta Java.

\section*{Abstract (English)}
\begin{otherlanguage}{british}
Over the last decade, quantum computing has been gaining prominence in the scientific field. With this, a full spectrum of tools have emerged focused on this new area and, among them, programming languages of a quantum nature.

At the same time, in a society as technological as ours, verifying the correct functioning of developed software has become an imperative need. In this way, a whole area of Software Engineering is dedicated to covering this need: \emph{Software Testing}.

Thus, the idea arises of taking one of the most important tools of testing, \textit{mutation testing}, and trying to transfer it to the quantum world in order to apply it to two of the most relevant quantum computing languages at the moment: $\qsh$ and \textit{Qiskit}.

To do this, a Java program has been developed that allows us to take code from both languages mentioned above and apply the most important techniques of \textit{mutation testing}. Furthermore, although in the current version it is only possible to execute code of $\qsh$ and \textit{Qiskit}, this program has been developed under the premise of being able to add other quantum languages in the future.

Throughout this work, the most important notions of quantum computing will be explained in detail, as well as the discipline of testing, with special emphasis on mutation testing. The planning followed for the development phase is also detailed, as well as the structure and internal functioning of the program. Finally, the execution of the program for one of the most important quantum algorithms is shown: the \textit{Deutsch-Jozsa algorithm.}
\end{otherlanguage}

\textbf{keywords}: Quantum computing, quantum testing, mutation testing, Microsoft \qsh, IBM Qiskit, Java tool.
