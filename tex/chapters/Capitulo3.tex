\chapter{Introducción al Testing}
Uno de los pilares fundamentales de este trabajo es el testing, más concretamente una técnica muy usada en este área, el \textit{mutation testing}. Por ello, a lo largo del siguiente capítulo daremos 
unas pinceladas muy básicas sobre en qué consiste el testing y hablaremos más en profundidad  de la técnica mencionada previamente.

\section{¿Qué es el Testing?}

El \emph{testing} es una disciplina de la \textit{Ingeniería del Software} orientada a aumentar la calidad y la fiabilidad de un programa. Su principal objetivo es encontrar errores en el software para de esta manera poder corregirlos.

Hay que tener muy en cuenta que demostrar el correcto funcionamiento de un programa es un problema no decidible, y por tanto este no es el objetivo del testing. En palabras de Edsger Wybe Dijkstra [\cite{dijkstra1972chapter}], el testing permite encontrar errores pero, sin embargo, no permite demostrar la ausencia de ellos.

A la hora de encontrar un fallo en el software, hay 3 términos que hay que tener muy claros para poder describir con precisión la naturaleza del fallo en cuestión. Se introducen los términos en inglés, pues su traducción puede inducir a confusiones no deseadas:
\begin{itemize}
\item \emph{Fault}: es el termino que se utiliza para indicar un defecto estático en el código del programa que está siendo testeado.
\item \emph{Error}: se corresponde con un estado incorrecto del programa durante la ejecución del mismo.
\item \emph{Failure}: se produce cuando el comportamiento externo del programa es incorrecto respecto a los requisitos establecidos para este.
\end{itemize}

La unidad u objeto fundamental del testing son los \emph{test}. Un test se constituye principalmente de dos componentes:
\begin{itemize}
\item \emph{Inputs}: son los valores que se necesita aportar al programa para que se ejecute de la forma concreta que está siendo testeada.
\item \emph{Outputs}: se corresponde con el resultado que producirá el programa en caso de que su comportamiento sea correcto.
\end{itemize}

Como el software puede constar de múltiples funcionalidades, va a ser necesario contar numerosos de test para poder verificarlo correctamente. A un conjunto de test se lo conoce como \textit{test suite}. 

Por otro lado, los programas suelen ser inmensamente extensos, y comprobarlos completamente no es viable. El número de inputs  que puede aceptar un programa puede ser ``infinito''. Pensemos por ejemplo en un compilador de Java. Potencialmente, los inputs que puede recibir el compilador no son sólo todos los programas Java, si no cualquier cadena de caracteres. La única limitación viene impuesta por el máximo tamaño de archivo que acepta el \textit{parser} del compilador. Por eso se dice que el número de inputs es ``infinito''. [\cite{ammann2016introduction}]

Siguiendo la literatura que acabamos de mencionar, la manera de solucionar este problema es introduciendo unos criterios que permitan al testeador diseñar test eficaces, es decir, tets que potencialmente sean más propensos a encontrar \textit{faults} en el código. Estos criterios son:
\begin{itemize}
\item \emph{Requisito de Test}: es un ``artefacto'' concreto del software que se debe cumplir durante el proceso de testing.
\item \emph{Requisitos de Cobertura}: son un conjunto de reglas que debe cumplir un \textit{test suite} para que se verifiquen todos los requisitos de test. Existen diversos tipos de cobertura basados en: grafos, lógica proposicional, dominio de los inputs ...
\end{itemize}

Una vez que se tienen unas nociones básicas del testing, podemos mover el foco hacia la rama del testing que realmente concierne a este trabajo. Igual que un \textit{test suite} nos va a permitir
encontrar fallos en un programa, es igual de interesante ver la eficiencia de dicho \textit{test suite}. Imaginemos que tenemos un programa con una serie de \emph{faults} en el código. Puede ocurrir que si se diseña un \textit{test suite} sobre unos requisitos de test pobres, dicho test suite no de lugar a ningún \emph{failure}, y esto no se debe a que el programa funcione correctamente,si no a que el conjunto de test no es bueno.

Así nace el \textit{mutation testing}, un técnica desarrollada con el objetivo de poder comprobar la eficacia de los test.

\section{Mutation Testing}

Las pruebas de mutación o \textit{mutation testing} fueron propuestas por Richard Lipton en 1971 [\cite{lipton1971fault}]. Esta técnica entra dentro de lo que se conoce como \textit{testing basado en sintaxis}, y como su propio nombre indica, el principal elemento a tener en cuenta en este tipo de testing es la propia sintaxis del lenguaje.

El objetivo primordial de las pruebas de mutación es comprobar la eficacia de un conjunto de test para un cierto programa. Para ello, se parte de un programa que funciona correctamente y se introducen pequeñas modificaciones en este para posteriormente correr los test sobre estos programas modificados y de esta forma comprobar si se siguen ejecutando con éxito.

Hay dos conceptos claves que nos permiten entender con mayor facilidad las pruebas de mutación:
\begin{definition}
Un \emph{operador de mutación} es una regla que especifica variaciones sintácticas para una cierta cadena en el lenguaje.
\end{definition}

\begin{definition}
Un \emph{mutante} es un nuevo programa que se obtiene al aplicar un operador de mutación sobre el programa original.
\end{definition}

Los operadores de mutación simulan errores típicos que puede cometer un programador. El primer paso a la hora de aplicar pruebas de mutación a un programa es definir los operadores de mutación para el lenguaje en el que esté escrito dicho programa. Por ejemplo, supongamos que tenemos el siguiente algoritmo:
\clearpage 

\begin{algorithm}
\caption{Ejemplo}
\begin{algorithmic}[1]
\Procedure{example}{$n, m$}\Comment{Devuelve $n$ si es estrictamente más grande que $m$.}
\If {$n > m$}
    \State  \textbf{return} $n$
\Else
    \State \textbf{return} $m$
\EndIf
\EndProcedure
\end{algorithmic}
\end{algorithm}


Un posible error que pueda cometer un programador es confundir los operadores de comparación: $>,\; <$. Por tanto, esta modificación sintáctica es un buen candidato a ser considerada como operador de mutación. Si aplicamos este operador al código anterior en la linea 2 se obtiene el siguiente mutante:

\begin{algorithm}
\caption{Mutante Ejemplo}
\begin{algorithmic}[1]
\Procedure{example}{$n, m$}\Comment{Devuelve $n$ si es estrictamente más grande que $m$.}
\If {$n < m$}
    \State  \textbf{return} $n$
\Else
    \State \textbf{return} $m$
\EndIf
\EndProcedure
\end{algorithmic}
\end{algorithm}

Es claro que el comportamiento de los dos algoritmos anteriores no es similar, y un buen conjunto de test debe ser capaz de diferenciar entre ambos. Cuando esto ocurre, decimos que el conjunto de test \textit{mata} al mutante. Con más detalle:

\begin{definition}
Sea $m$ un mutante obtenido al aplicar un operador de mutación a cierto programa $p$.
Decimos que un test $t$ \emph{mata} a $m$ si $p$ y $m$ producen un output distinto al aplicar los inputs dados por $t$.
\end{definition}

De la misma forma, dado un conjunto de test $\mathcal{T}$, diremos que $\mathcal{T}$ \textit{mata} a un mutante $m$ si $\exists t\in \mathcal{T},,\; t \; mata \;a\; m$. Ese caso, diremos que el mutante $m$ \emph{muere}, y por el contrario, si ningún test del conjunto de test mata al mutante, diremos que $m$ \emph{vive}.

Esta definición se corresponde con el criterio conocido como \textit{strong mutation testing}. Al final de la sección se verá la diferencia con otros criterios.

Por tanto, volviendo a los ejemplos, si ejecutamos el programa original con los valores $(6,5)$ la función devolverá el valor $6$. Sin embargo, aplicando el mismo par de valores al mutante, el programa mutado devuelve el valor $5$. Por lo tanto, el test formado por el par de inputs $(6,5)$ \emph{mata} a dicho mutante.

Pero, ¿qué ocurre si un conjunto de test no mata a un cierto mutante? Esto puede ser debido principalmente a dos causas:
\begin{itemize}
\item Generalmente esto ocurre si el conjunto de test no contiene alguno de los test que potencialmente mata al mutante.
\item Se puede dar que no exista test que nos permita distinguir entre el mutante y el programa original en función de sus outputs. En ese caso diremos que se tiene un  \emph{mutante equivalente.}
\end{itemize}

%Aquí habría que citar a Manolo, pero no sé si va a querer que le cite o no%
Nos interesa tener una métrica que nos permita saber como de eficiente es cierto conjunto de test $\mathcal{T}$ a la hora de matar un conjunto de mutantes $\mathcal{M}$ obtenidos a raíz de un programa $p$. Denotemos al subconjunto de mutantes que es matado por $\mathcal{T}$ como $\mathcal{M}^*$. Por tanto, parece sensato definir una métrica de eficiencia en función de la fracción de mutantes que mata nuestro conjunto de test:
\begin{center}
$E(\mathcal{T},\mathcal{M},p) = \dfrac{|\mathcal{M}^*|}{|\mathcal{M} \,|}$, donde $|\cdot|$ denota el cardinal del conjunto
\end{center}

Sin embargo, tal y como se ha mencionado previamente, es imposible matar a un mutante equivalente, luego para mejor la precisión de la métrica, es necesario excluir el conjunto de mutantes equivalentes $\mathcal{M}_e$. Se obtiene así una nueva métrica más ajustada que la anterior:

%Añado un espaciado horizontal para que quede con el mismo espacio que la metrica anterior%
$\hspace{48pt}E^+(\mathcal{T},\mathcal{M},p) = \dfrac{|\mathcal{M}^*|}{|\mathcal{M} \,|\setminus|\mathcal{M}_e|}$

Si bien esto sería lo ideal, identificar los mutantes equivalentes es un problema indecible, y por tanto nos tendremos que conformar con la métrica $E$. Evitar la generación de un número elevado de mutantes equivalentes sigue siendo uno de los problemas a abordar en mutation testing. 

Otro problema importante es el elevado número de mutantes que se generan, ya que incluso en programas pequeños, en un lenguaje con pocos operadores de mutación definidos,el número de mutantes obtenidos puede ser inmensamente grande. Una solución que ayuda a mitigar este problema es trabajar sobre lo que se conoce como la \textit{hipótesis del programador competente} [\cite{demillo1978hints}]. Esta hipótesis tiene dos puntos clave:
\begin{itemize}
\item La mayor parte de los errores en el software cometidos por un programador senior son debidos a pequeños errores sintácticos.
\item Un programador avanzado no suele más de una vez este tipo de error. Por lo tanto no se aplicará más de un operador de mutación a la hora de generar un mutante.
\end{itemize}

Otro de las hipótesis sobre las que se suele trabajar en mutation testing es el conocido \textit{efecto de acoplamiento}. Esta hipótesis propone que los errores más graves en un programa están \textit{acoplados} sobre errores sencillos,y por tanto si un test es capaz de desvelar estos errores sencillos, entonces también desvelará los graves [\cite{offutt1992investigations}].

Por último, tal y como se ha mencionado previamente, la forma de matar un mutante es comparando los outsputs del programa. Esto se conoce como \textit{strong mutation testing}. Sin embargo, no es el único criterio que se puede considerar a la hora de decidir si un mutante muere o no. Veamos con precisión los dos métodos más utilizados:

\begin{definition}
Sea $p$ un programa,$t$ un test y $m$ un mutante obtenido a partir de la aplicación de un operador de mutación sobre $p$. Podemos decir que $t$ mata a $m$ en función de dos criterios:
\begin{itemize}
\item \emph{Strong Mutation Testing}: $t$ mata a $m$ si los outputs obtenidos por $p$ y $m$ al aplicar los inputs de $t$ son distintos.
\item \emph{Strong Mutation Testing}: $t$ mata a $m$ si, usando los inputs dados por $t$,  tras ejecutar la sentencia mutada para $m$ y la sentencia original para $p$ el estado interno del programa es distinto. 
\end{itemize}
\end{definition}

Véase que este ultimo criterio, al ser menos restrictivo, ayuda a matar más mutantes. Esto se debe a que un mutante $m$ y un programa $p$ pueden tener estados internos distintos durante la ejecución del programa, y sin embargo producir el mismo output. 

