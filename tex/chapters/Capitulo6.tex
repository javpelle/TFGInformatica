\chapter{Ejemplos y conclusiones}

\section{Pruebas de mutación sobre Deutsch-Jozsa}

Vamos a dar uso a MTQC. Implementaremos el algoritmo de \textit{Deutsch-Jozsa} [\cite{deutsch1992rapid}] en los lenguajes \textit{Qiskit} y \qsh\ y aplicaremos pruebas de mutación sombre ambos códigos. Antes veremos que problema plantea dicho algoritmo.

Sea $\function{f}{\{0,1\}^n}{\{0,1\}}$ una función binaria que bien puede ser constante ($f(x) = 0$ o $f(x) = 1$ para todo $x\in\{0,1\}^n$) o bien es balanceada (la salida es 0 para la mitad de entradas y 1 para la otra mitad). Desde el punto de vista clásico, en el caso peor hay que verificar $2^{n-1}+1$ entradas para resolver el problema. El algoritmo cuántico de \textit{Deutsch-Jozsa} lo resuelve en una sola iteración.

Previamente a mostrar el código de nuestro ejemplo particular, analicemos en detalle cada paso del que consta el algoritmo de Jozsa.

En primer lugar, si nuestra función $f$ toma valores en $\{0,1\}^n$, debemos crear un circuito con $n + 1$ qubits. Todos los qubits deberan tomar el estado $\ket{0}$ menos el último de ellos, que utilizaremos como auxiliar e inicializamos al estado $\ket{1}$. De esta forma nuestro sistema se encuentra en un primer estado dado por:
\begin{equation}
\ket{\phi_0} = \ket{0\overset{n}{\cdots}01}
\end{equation}

A continuación, aplicamos una puerta Hadamard a cada qubit, obteniendo de esta forma el siguiente estado:
\begin{equation}
\ket{\phi_1} = \dfrac{1}{\sqrt{2^{n+1}}} \sum\limits_{i=0}^{2^n - 1} \ket{i}(\ket0-\ket1) \textrm{, donde $i$ se corresponde con su representación binaria.}
\end{equation}

Posteriormente, hay que aplicar el operador $U_f$ tal y como fue definido en \ref{uf}. Si denotamos por $x$ a los $n$ qubits de entrada y por $y$ al qubit auxiliar se tiene:
\begin{equation}
U_f\colon \begin{matrix}x&\longrightarrow& x\\ \y&\longrightarrow& y\oplus f(x)\end{matrix}
\end{equation}

El diseño de esta $U_f$ puede ser muy complejo, dependiendo de cómo sea la función $f$. En nuestro ejemplo hemos escogido una función que facilita la creación del operador $U_f$.

Tras aplicar el operador $U_f$, se obtiene un tercer estado dado por:
\begin{equation}
\ket{\phi_2} = \dfrac{1}{\sqrt{2^{n+1}}} \sum\limits_{i=0}^{2^n - 1} \ket{i}(\ket{f(i)}-\ket{1\oplus f(i)}) 
\end{equation}

Ahora, como $f(i)$ sólo puede tomar valores binarios, podemos simplificar la expresión anterior:
\begin{equation}
\ket{\phi_2} = \dfrac{1}{\sqrt{2^{n+1}}} \sum\limits_{i=0}^{2^n - 1} (-1)^{f(i)}\ket{i}(\ket0-\ket1) 
\label{phi2}
\end{equation}


A partir de este punto, el estado del último qubit $\dfrac{1}{\sqrt{2}}(\ket0-\ket1)$ puede ser ignorado, pues no será relevante para el resto de cálculos.

A continuación, debemos aplicar una puerta Hadamard a cada uno de los $n$ qubits restantes. Previamente a realizar dicha operación, vamos a ver cómo una representación matemática de la aplicación simultanea de puertas Hadamard que facilita la comprensión de la parte final del algoritmo.

Cuando contamos con un sólo qubit, una puerta Hadamard sabemos que viene dada por: \begin{center}
$\gatetwo{H}{\dfrac{1}{\sqrt{2}}(\ket0+\ket1)}{\dfrac{1}{\sqrt{2}}(\ket0-\ket1)}$
\end{center}

Por tanto, podemos escribir el caso general, donde $x=0$ o $x=1$ como:
\begin{equation}
H\ket{x} = \tfrac{1}{\sqrt{2}}\sum\limits_{z\in\{0,1\}} (-1)^{x\cdot z}\ket{z} \textrm{, donde $x\cdot z$ denota el producto escalar bit a bit, módulo 2}
\end{equation}

Veamos ahora el caso con 2 qubits a los que aplicamos una puerta Hadamard a cada uno:
\begin{equation}
\begin{split}
H^{\otimes 2}\ket{x_1,x_2} & = H\ket{x_1} \otimes H\ket{x_2} \\
& = \tfrac{1}{\sqrt{2}}\sum\limits_{z_1\in\{0,1\}} (-1)^{x_1\cdot z_1}\ket{z_1} \otimes \tfrac{1}{\sqrt{2}}\sum\limits_{z_2\in\{0,1\}} (-1)^{x_2\cdot z_2}\ket{z_1} \\
& = \tfrac{1}{\sqrt{2^2}}\sum\limits_{z_1,z_2\in\{0,1\}} (-1)^{x_1\cdot z_1+x_2\cdot z_2}\ket{z_1,z_2}
\end{split}
\end{equation}

De esta misma forma, podemos representar la aplicación de puertas Hadamard a $n$ qubits como:
\begin{equation}
\begin{split}
H^{\otimes n}\ket{x_1,\cdots,x_n} & = \tfrac{1}{\sqrt{2^n}}\sum\limits_{z_1,\cdots,z_n\in\{0,1\}} (-1)^{x_1\cdot z_1+\cdots+x_n\cdot z_n}\ket{z_1,\cdots,z_n}
\end{split}
\end{equation}

Si retomamos ahora la notación para el sumatorio utilizada a lo largo de la demostración, podemos expresar el resultado anterior como:
\begin{equation}
\begin{split}
H^{\otimes n}\ket{x_1,\cdots,x_n} & = \tfrac{1}{\sqrt{2^n}}\sum\limits_{z=0}^{2^n-1} (-1)^{x\cdot z}\ket{z}
\end{split}
\end{equation}

Por tanto, partiendo de la expresión \ref{phi2}, si aplicamos una puerta Hadamard a los $n$ primeros qubits (recordemos que el estado del último qubit ya no era relevante),
se obtiene un nuevo estado $\phi_3$ dado por:

\begin{equation}
\begin{split}
\ket{\phi_3} & = \dfrac{1}{\sqrt{2^{n}}} \sum\limits_{i=0}^{2^n - 1} (-1)^{f(i)}\left[\dfrac{1}{\sqrt{2^{n}}}\sum\limits_{j=0}^{2^n - 1}(-1)^{i\cdot j}\ket{j}\right] \\
& = \dfrac{1}{2^n}\sum\limits_{j=0}^{2^n - 1}\left[\sum\limits_{i=0}^{2^n - 1}(-1)^{f(i)}(-1)^{i\cdot j}\right]\ket{j}
\end{split}
\label{phi3}
\end{equation}

Una vez que hayamos obtenido este estado, vamos a estudiar cual es la probabilidad de obtener el estado $\ket{0\overset{n}{\cdots}0}$ al medir los $n$ primeros qubits.

En virtud de (\ref{phi3}), el cuadrado de la amplitud del estado $\ket{0\overset{n}{\cdots}0}$ viene dada por:

\begin{equation}
\begin{split}
& \left|\dfrac{1}{2^n}\sum\limits_{i=0}^{2^n - 1}(-1)^{f(i)}(-1)^{i\cdot j}\right|^2 \\ 
= &\left|\dfrac{1}{2^n}\sum\limits_{i=0}^{2^n - 1}(-1)^{f(i)}\right|^2 \textrm{, ya que j = $0\overset{n}{\cdots}0$}
\end{split}
\end{equation}

Por tanto, si la \emph{función $f$ es balanceada}, la mitad de los términos $(-1)^{f(i)}$ evaluaran a $1$, mientras que la otra mitad a $-1$, otorgando por tanto \emph{probabilidad 0} de obtener el estado 
$\ket{0\overset{n}{\cdots}0}$.

Si por el contrario la \emph{función $f$ es constante}, entonces todos los términos $(-1)^{f(i)}$ tendrán el mismo signo, no cancelándose entre ellos, y por tanto obteniendo \emph{probabilidad 1} de medir el estado $\ket{0\overset{n}{\cdots}0}$.

De esta forma, se puede discernir de manera completamente determinista si la función $f$ es constante o balanceada.

\begin{figure}[!htb]
\[\Qcircuit @C=1em @R=.7em {
\lstick{\ket{0}} & \qw & \gate{H} & \qw &\multigate{4}{U_f}&\qw& \gate{H} & \qw & \meter & \qw \\
\lstick{\ket{0}} & \qw & \gate{H} & \qw & \ghost{U_f}      &\qw& \gate{H} & \qw & \meter & \qw \\
\lstick{\ket{0}} & \qw & \gate{H} & \qw & \ghost{U_f}      &\qw& \gate{H} & \qw & \meter & \qw \\
\lstick{\ket{0}} & \qw & \gate{H} & \qw & \ghost{U_f}      &\qw& \gate{H} & \qw & \meter & \qw \\
\lstick{\ket{1}} & \qw & \gate{H} & \qw & \ghost{U_f}      &\qw& \qw      & \qw & \qw    & \qw}\]
\caption{Circuito de una implementación de Deutsch-Jozsa.}
\label{fig:fig61}
\end{figure}

En la figura \ref{fig:fig61} podemos ver el circuito que resume todo lo contado anteriormente. Una vez que hemos visto cuales son los pasos a aplicar para ejecutar el algoritmo de Deutsch-Jozsa, se procede a exponer el código de dicho algoritmo para los lenguajes $\qsh$ y \textit{Qiskit} además de las funciones $f$ utilizadas. Empecemos por estas últimas.

Emplearemos 3 funciones $\{0,1\}^4\longrightarrow\{0,1\}$ que vienen definidas por:
\begin{itemize}
\item $f_1(x_1,x_2,x_3,x_4)=\left\{\begin{matrix}1 \mathrm{\ si\ } x_1=1\\0 \mathrm{\ si\ } x_1=0\end{matrix}\right.$ por lo que se trata de una función balanceada.

\item $f_2(x_1,x_2,x_3,x_4)=0$ por lo que se trata de una función constante.

\item $f_3(x_1,x_2,x_3,x_4)=1$ por lo que se trata de una función constante.
\end{itemize}

Las $U_f$ respectivas son fáciles de implementar. En el primer caso basta con aplicar una puerta CNOT empleando como controlador el qubit correspondiente al bit $x_1$ y como qubit receptor el de salida. En el caso de $U_{f_2}$ no hay que realizar ninguna operación, aunque pondremos simbólicamente la puerta identidad aplicada al qubit de salida. Por último, $U_{f_3}$ viene determinada por una puerta $X$ aplicada al qubit de salida.

\begin{figure}[!htb]
\begin{lstlisting}[language=Python]
def deutschjozsa (qc, qr, cr, uf):
    if len(qr) != 5:
        print("El numero de qubits debe ser 5.")
        return
    
    # Negacion  del ultimo qubit
    qc.x(qr[-1])
    
    # Aplicacion de una puerta de Hadamard a cada qubit
    for r in qr:
        qc.h(r)
        
    # Aplicamos U_f
    uf(qc, qr)
        
    # Aplicacion de una puerta de Hadamard a cada qubit de entrada
    for r in qr[:-1]:
        qc.h(r)
            
    qc.measure(qr[:-1], cr)

def uf_1 (qc, qr):
    # f(x_1,x_2,x_3,x_4) = 1 si x_1 vale 1, 0 en otro caso. Balanceada
    # U_f cnot, como control el primer cubit, X aplicada a qubit de salida
    qc.cx(qr[0], qr[-1])
    
def uf_2 (qc, qr):
    # f(x_1,x_2,x_3,x_4) = 0. Constante
    # U_f Identidad qubit de salida
    qc.iden(qr[-1])
    
def uf_3 (qc, qr):
    # f(x_1,x_2,x_3,x_4) = 1. Constante
    # U_f not en qubit de salida
    qc.x(qr[-1])
\end{lstlisting}
\caption{Código en \textit{Qiskit} de una implementación de Deutsch-Jozsa.}
\label{fig:code62}
\end{figure}

En la figura \ref{fig:code62} podemos ver el código en \textit{Qiskit} de la implementación del algoritmo de \textit{Deutsch-Jozsa} para un sistema de 5 qubits y cada una de las funciones $U_f$ expresadas anteriormente.

A continuacón mostramos el código en \qsh de la implementación del mismo algoritmo: 


\begin{figure}[htb]
\begin{lstlisting}[language=c++]
operation DeutschJozsa(register : Qubit[], U_f : (Qubit[] => Unit)) : String {  
    let nQubits = Length(register);
    if (nQubits != 5) {
   		return "El numero de Qubits debe ser 5 ";
    }
    else {
        //Negacion del ultimo Qubit (salida)
        X(register[nQubits - 1]); 
        //Poner cada qubit en superposicion
        for(q in register) {
        	H(q);
        }         
        //Aplicamos la Uf
        U_f(register);
        //Aplicamos una puerta Hadamard a cada Qubit de entrada
        for(i in 0..nQubits - 2) {
        	H(register[i]);
        }        
        //La funcion sera constante si medimos el estado 0000. Balanceada en otro caso.
        mutable allZeros = true;
        for(i in 0..nQubits - 2) {
           	if(M(register[i]) == One){
            	set allZeros = false;
            }
        } 
        if (allZeros){
        	return "Constante";
        } else {
       		return "Balanceada";
       }
     }  
}
operation uf(register : Qubit[]) : Unit{
		//f(x_1,x_2,x_3,x_4) = 1 si x_1 vale 1, 0 en otro caso. Balanceada
		CNOT(register[0], register[Length(register) - 1]);
}

operation uf2(register : Qubit[]) : Unit{
		//f(x_1,x_2,x_3,x_4) = 1. Constante
		I(register[Length(register) - 1]);
}

operation uf3(register : Qubit[]) : Unit { 
		//f(x_1,x_2,x_3,x_4) = 0. Constante  
		X(register[Length(register) - 1]);
   }
\end{lstlisting}
\caption{Código en \qsh de una implementación de Deutsch-Jozsa.}
\end{figure}

\section{Implementaciones futuras}

Queremos aportar algunas ideas con las que se puede dar continuidad a este proyecto. Algunas de estas funcionalidades fueron planificadas al principio del proyecto para ser descartadas más adelante y otras surgen durante la implementación de MTQC.

La implementación de \textit{weak mutation testing} podría ser una buena alternativa a los test ya existentes en el programa. Se trataría de un test determinista a realizar sobre simulador que verificara las condiciones de la definición \ref{def:def24}. También podría añadirse algún lenguaje adicional como \textbf{OpenQASM}.

Por otro lado sería conveniente darle más facilidades al usuario para probar sus programas como permitir que puedan añadirse nuevos operadores de mutación durante la ejecución y que estos puedan ser guardados para futuras pruebas de mutación o guardar el proceso en un determinado momento para que la tarea pueda ser retomada en otra ejecución. Además, sería conveniente facilitar el uso de otros archivos y librerías externas para evitar, como mencionamos antes, que el usuario no tenga que realizar una adición de código para encontrar la ruta a estos archivos.

Por último, sería conveniente habilitar una opción para que un mutante pueda tener 2 o más mutaciones y mejorar la eficiencia del programa mediante la paralelización de cada una de las ejecuciones realizadas al aplicar los casos de prueba.


