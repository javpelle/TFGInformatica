\chapter{Ejemplos y conclusiones}

\section{Implementaciones futuras}

Queremos aportar algunas ideas con las que se puede dar continuidad a este proyecto. Algunas de estas funcionalidades fueron planificadas al principio del proyecto para ser descartadas más adelante y otras surgen durante la implementación de MTQC.

La implementación de \textit{weak mutation testing} podría ser una buena alternativa a los test ya existentes en el programa. Se trataría de un test determinista a realizar sobre simulador que verificara las condiciones de la definición \ref{def:def24}. También podría añadirse algún lenguaje adicional como \textbf{OpenQASM}.

Por otro lado sería conveniente darle más facilidades al usuario para probar sus programas como permitir que puedan añadirse nuevos operadores de mutación durante la ejecución y que estos puedan ser guardados para futuras pruebas de mutación o guardar el proceso en un determinado momento para que la tarea pueda ser retomada en otra ejecución. Además, sería conveniente facilitar el uso de otros archivos y librerías externas para evitar, como mencionamos antes, que el usuario no tenga que realizar una adición de código para encontrar la ruta a estos archivos.

Por último, sería conveniente habilitar una opción para que un mutante pueda tener 2 o más mutaciones y mejorar la eficiencia del programa mediante la paralelización de cada una de las ejecuciones realizadas al aplicar los casos de prueba.


