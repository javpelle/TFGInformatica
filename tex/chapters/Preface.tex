
\chapter*{Prefacio}

En esta memoria queremos reflejar las nociones básicas de las pruebas de mutación aplicadas a la computación cuántica. Por nuestra condición de estudiantes del doble grado de Matemáticas e Ingeniería Informática, la computación y teoría de la información cuántica representa un área de confluencia de las Matemáticas, la Física y la Informática muy apropiada para abordar en un TFG.
%
En total se han desarrollado 3 Trabajos de Fin de Grado en relación a este tema: dos realizados para la Facultad de Matemáticas, por cada uno de los integrantes de este grupo, y este mismo. Haremos referencia a este hecho durante la memoria pues, debido al tema común de los trabajos, tareas como la investigación y la planificación se elaboran de manera conjunta.
%
Queremos añadir que creemos que este proyecto sería más completo y {\it redondo} si se hubiera combinado en un único TFG tal y como se venía haciendo hasta el curso pasado. Tanto los conocimientos reflejados como la extensión de esta posible combinación deberían ser suficientes para satisfacer los requisitos que proponen cualquiera de las dos facultades. Con el modo actual, cada investigación por separado pierde valor y se hace más complicado reflejar todos los contenidos abarcados y lo más importante, la conexión entre ellos.
%
Nos hemos esforzado para que esta memoria sea autocontenida y no requiera de otros documentos, pero si el lector tiene interés, recomendamos la lectura de nuestras memorias presentadas como TFGs en la Facultad de  Matemáticas porque conforman un pilar sólido para obtener una  introducción a la computación cuántica.

Consideramos que los prerrequisitos para leer esta memoria se restringen a nociones realmente básicas de álgebra lineal y computación clásica. Presentamos los elementos necesarios, a un nivel fundamental, en las tres grandes áreas que se consideran en este TFG (computación cuántica, \textit{mutation testing} y \textit{mutation testing} aplicado a la computación cuántica), no sin antes realizar una introducción breve sobre los antecedentes de estas materias y algunas referencias actuales.

Una parte vital de la investigación que presentamos en esta memoria es el proyecto de desarrollo de \textit{software} que hemos llamado \textit{Mutation Testing for Quantum Computing} (MTQC). Destinamos un capítulo a narrar su gestión, planificación, desarrollo y funcionamiento. El programa principal en el que se apoya esta TFG está escrito en \textit{Java} aunque también se utilizan otros lenguajes como \textit{Python}, \textit{Qiskit} y \qsh. Hemos procurado que su desarrollo sea lo más modular posible para que pueda tener continuidad como proyecto al ser relativamente fácil implementar nuevos lenguajes y funcionalidades.
%
En la parte final de esta memoria realizamos pruebas de mutación, utilizando MTQC sobre programas cuánticos escritos en \textit{Qiskit} y \qsh, y establecemos algunas conclusiones sobre la investigación y el proyecto, además de aportar algunas ideas que mejorarían MTQC y que podrían ser desarrolladas en un futuro cercano.

Queríamos agradecer a nuestro tutores Mercedes G. Merayo y Manuel Núñez la proposición de un tema tan interesante y completo para elaborar nuestro TFG. Si bien nos ha hecho invertir mucho tiempo y tal vez otro tema hubiera resultado más sencillo y breve, nos gustaría pensar que estos conocimientos  puedan ser una punta de lanza, en el mundo académico y laboral, en un futuro cercano si la tecnología permite el desarrollo de computadores cuánticos cada vez más potentes. Por último, nos gustaría agradecer a nuestros padres la inversión y el apoyo para poder realizar nuestra formación académica universitaria y, en general, por aguantarnos.