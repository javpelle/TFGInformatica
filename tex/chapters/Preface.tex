\chapter*{Prefacio}

En esta memoria queremos reflejar las nociones básicas de las pruebas de mutación aplicadas a la computación cuántica. Por nuestra condición de estudiantes del doble grado de Matemáticas e Ingeniería Informática, la computación y teoría de la información cuántica representa un área de confluencia de las matemáticas, la física y la informática muy apropiada para abordar en un TFG.

En total se han desarrollado por nuestra parte 3 Trabajos de Fin de Grado en relación a este tema: dos realizados para la facultad de Matemáticas por cada uno de los integrantes de este grupo y este mismo. Haremos referencia de este hecho durante la memoria pues, debido al tema común de los trabajos, tareas como la investigación y la planificación se elaboran de manera conjunta.

Queremos añadir que creemos que este proyecto en concreto sería más completo si se hubiera combinado en un único TFG tal y como se venía haciendo hasta el curso pasado. Tanto los conocimientos reflejados como la extensión de esta posible combinación deberían ser suficientes para satisfacer los requisitos que proponen cualquiera de las dos facultades. Con el modo actual, cada investigación por separado pierde valor y se hace más complicado reflejar todos los contenidos abarcados y lo más importante, la conexión entre ellos.

Nos hemos esforzado porque esta memoria sea suficiente por sí misma y no requiera de otras lecturas, pero si el lector tiene interés, recomendamos la lectura de cualquiera de las memorias realizadas por nosotros como TFG de Matemáticas que forman un pilar sólido de la introducción a la computación cuántica aquí mencionada.

En cuanto a esta memoria, los prerrequisitos para su comprensión son nociones realmente básicas en álgebra lineal y computación clásica. Introducimos todos los elementos necesarios desde un nivel fundamental: computación cuántica, \textit{testing} y \textit{mutation testing} y \textit{mutation testing} aplicado a la computación cuántica; no sin antes realizar una introducción breve sobre los antecedentes de estas materias y algunas referencias de actualidad.

Una parte vital de esta investigación es el el proyecto de \textit{software} que hemos llamado \textit{Mutation Testing for Quantum Computing} (MTQC). Destinamos un capítulo a narrar su gestión, planificación, desarrollo y funcionamiento. El programa principal está escrito en \textit{Java} aunque otros lenguajes son utilizados como \textit{Python} y los cuánticos \textit{Qiskit} y \qsh. Hemos procurado que su desarrollo sea lo más modular posible para que pueda tener continuidad como proyecto al ser relativamente fácil implementar nuevos lenguajes y funcionalidades.

En la parte final de esta memoria realizamos algunas pruebas de mutación utilizando MTQC sobre programas cuánticos escritos en \textit{Qiksit} y \qsh\ y establecemos algunas conclusiones sobre la investigación y el proyecto además de aportar algunas ideas que mejorarían MTQC y que podrían ser construidas en un futuro cercano.

Queríamos agradecer a nuestro tutor Manuel Núñez la proposición de un tema tan interesante y completo para elaborar nuestro TFG. Si bien nos ha hecho invertir mucho tiempo y tal vez otro tema hubiera resultado más sencillo y breve, estos conocimientos quizás puedan ser la punta de lanza en el mundo académico y laboral en un futuro inmediato si la tecnología permite el desarrollo de computadores cuánticos cada vez más capaces. Por último, agradecer a nuestros padres la inversión y el apoyo para poder realizar nuestra formación académica universitaria y, en general, por aguantarnos.